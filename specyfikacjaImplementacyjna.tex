\documentclass[10pt, oneside]{article}

\usepackage[T1]{fontenc}
\usepackage[utf8]{inputenc}
\usepackage{polski}
\usepackage{indentfirst}
\usepackage{caption}
\usepackage{float}
\usepackage{tikz}
\usepackage{polski}
\usepackage{fancyhdr}
\usepackage{lastpage}
\usepackage{tcolorbox}
\usepackage{graphicx}

\title{Specyfikacja implementacyjna automatu komórkowego - WireWorld, Game of Life}
\author{Danuta Stawiarz, Katarzyna Stankiewicz}
\date{21 maja 2019 r.}

\pagestyle{fancy}
\fancyhf{} 
\lhead{}
\rhead{} 
	
\rfoot{
\begin{center} Strona \thepage \hspace{1pt} z \pageref{LastPage}
\end{center}
}
%____________________________________________________________________

\begin{document}
\maketitle
\tableofcontents
\newpage	

\section{Projekt systemu}
\subsection{Wzorzec projektowy}
W projekcie zastosowany będzie wzorzec MVC (Model View Controller). Aby ułatwić pracę nad projektem oraz późniejsze modyfikacje, wprowadzony zostanie przejrzysty przydział poszczególnych plików do odpowiednich pakietów. 
W pakiecie models będą się znajdować klasy, które służą do wykonywania wszelkich operacji związanych z implementacją funkcjonalności automatu komórkowego.
W pakiecie View znajdą się wszystkie pliki .fxml odpowiedzialne za wygląd automatu komórkowego.
W pakiecie controllers będą zawarte kontrolery, służące do obsługi żądań użytkownika.

\subsection{Diagram modułów}

\subsection{Przepływ sterowania}

\subsection{Zastosowane algorytmy}


\section{Testy}




\end{document}